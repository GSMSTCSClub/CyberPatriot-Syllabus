% Options for packages loaded elsewhere
\PassOptionsToPackage{unicode}{hyperref}
\PassOptionsToPackage{hyphens}{url}
\PassOptionsToPackage{dvipsnames,svgnames,x11names}{xcolor}
%
\documentclass[
  letterpaper,
  DIV=11,
  numbers=noendperiod]{scrartcl}

\usepackage{amsmath,amssymb}
\usepackage{lmodern}
\usepackage{iftex}
\ifPDFTeX
  \usepackage[T1]{fontenc}
  \usepackage[utf8]{inputenc}
  \usepackage{textcomp} % provide euro and other symbols
\else % if luatex or xetex
  \usepackage{unicode-math}
  \defaultfontfeatures{Scale=MatchLowercase}
  \defaultfontfeatures[\rmfamily]{Ligatures=TeX,Scale=1}
\fi
% Use upquote if available, for straight quotes in verbatim environments
\IfFileExists{upquote.sty}{\usepackage{upquote}}{}
\IfFileExists{microtype.sty}{% use microtype if available
  \usepackage[]{microtype}
  \UseMicrotypeSet[protrusion]{basicmath} % disable protrusion for tt fonts
}{}
\makeatletter
\@ifundefined{KOMAClassName}{% if non-KOMA class
  \IfFileExists{parskip.sty}{%
    \usepackage{parskip}
  }{% else
    \setlength{\parindent}{0pt}
    \setlength{\parskip}{6pt plus 2pt minus 1pt}}
}{% if KOMA class
  \KOMAoptions{parskip=half}}
\makeatother
\usepackage{xcolor}
\setlength{\emergencystretch}{3em} % prevent overfull lines
\setcounter{secnumdepth}{5}
% Make \paragraph and \subparagraph free-standing
\ifx\paragraph\undefined\else
  \let\oldparagraph\paragraph
  \renewcommand{\paragraph}[1]{\oldparagraph{#1}\mbox{}}
\fi
\ifx\subparagraph\undefined\else
  \let\oldsubparagraph\subparagraph
  \renewcommand{\subparagraph}[1]{\oldsubparagraph{#1}\mbox{}}
\fi

\usepackage{color}
\usepackage{fancyvrb}
\newcommand{\VerbBar}{|}
\newcommand{\VERB}{\Verb[commandchars=\\\{\}]}
\DefineVerbatimEnvironment{Highlighting}{Verbatim}{commandchars=\\\{\}}
% Add ',fontsize=\small' for more characters per line
\usepackage{framed}
\definecolor{shadecolor}{RGB}{241,243,245}
\newenvironment{Shaded}{\begin{snugshade}}{\end{snugshade}}
\newcommand{\AlertTok}[1]{\textcolor[rgb]{0.68,0.00,0.00}{#1}}
\newcommand{\AnnotationTok}[1]{\textcolor[rgb]{0.37,0.37,0.37}{#1}}
\newcommand{\AttributeTok}[1]{\textcolor[rgb]{0.40,0.45,0.13}{#1}}
\newcommand{\BaseNTok}[1]{\textcolor[rgb]{0.68,0.00,0.00}{#1}}
\newcommand{\BuiltInTok}[1]{\textcolor[rgb]{0.00,0.23,0.31}{#1}}
\newcommand{\CharTok}[1]{\textcolor[rgb]{0.13,0.47,0.30}{#1}}
\newcommand{\CommentTok}[1]{\textcolor[rgb]{0.37,0.37,0.37}{#1}}
\newcommand{\CommentVarTok}[1]{\textcolor[rgb]{0.37,0.37,0.37}{\textit{#1}}}
\newcommand{\ConstantTok}[1]{\textcolor[rgb]{0.56,0.35,0.01}{#1}}
\newcommand{\ControlFlowTok}[1]{\textcolor[rgb]{0.00,0.23,0.31}{#1}}
\newcommand{\DataTypeTok}[1]{\textcolor[rgb]{0.68,0.00,0.00}{#1}}
\newcommand{\DecValTok}[1]{\textcolor[rgb]{0.68,0.00,0.00}{#1}}
\newcommand{\DocumentationTok}[1]{\textcolor[rgb]{0.37,0.37,0.37}{\textit{#1}}}
\newcommand{\ErrorTok}[1]{\textcolor[rgb]{0.68,0.00,0.00}{#1}}
\newcommand{\ExtensionTok}[1]{\textcolor[rgb]{0.00,0.23,0.31}{#1}}
\newcommand{\FloatTok}[1]{\textcolor[rgb]{0.68,0.00,0.00}{#1}}
\newcommand{\FunctionTok}[1]{\textcolor[rgb]{0.28,0.35,0.67}{#1}}
\newcommand{\ImportTok}[1]{\textcolor[rgb]{0.00,0.46,0.62}{#1}}
\newcommand{\InformationTok}[1]{\textcolor[rgb]{0.37,0.37,0.37}{#1}}
\newcommand{\KeywordTok}[1]{\textcolor[rgb]{0.00,0.23,0.31}{#1}}
\newcommand{\NormalTok}[1]{\textcolor[rgb]{0.00,0.23,0.31}{#1}}
\newcommand{\OperatorTok}[1]{\textcolor[rgb]{0.37,0.37,0.37}{#1}}
\newcommand{\OtherTok}[1]{\textcolor[rgb]{0.00,0.23,0.31}{#1}}
\newcommand{\PreprocessorTok}[1]{\textcolor[rgb]{0.68,0.00,0.00}{#1}}
\newcommand{\RegionMarkerTok}[1]{\textcolor[rgb]{0.00,0.23,0.31}{#1}}
\newcommand{\SpecialCharTok}[1]{\textcolor[rgb]{0.37,0.37,0.37}{#1}}
\newcommand{\SpecialStringTok}[1]{\textcolor[rgb]{0.13,0.47,0.30}{#1}}
\newcommand{\StringTok}[1]{\textcolor[rgb]{0.13,0.47,0.30}{#1}}
\newcommand{\VariableTok}[1]{\textcolor[rgb]{0.07,0.07,0.07}{#1}}
\newcommand{\VerbatimStringTok}[1]{\textcolor[rgb]{0.13,0.47,0.30}{#1}}
\newcommand{\WarningTok}[1]{\textcolor[rgb]{0.37,0.37,0.37}{\textit{#1}}}

\providecommand{\tightlist}{%
  \setlength{\itemsep}{0pt}\setlength{\parskip}{0pt}}\usepackage{longtable,booktabs,array}
\usepackage{calc} % for calculating minipage widths
% Correct order of tables after \paragraph or \subparagraph
\usepackage{etoolbox}
\makeatletter
\patchcmd\longtable{\par}{\if@noskipsec\mbox{}\fi\par}{}{}
\makeatother
% Allow footnotes in longtable head/foot
\IfFileExists{footnotehyper.sty}{\usepackage{footnotehyper}}{\usepackage{footnote}}
\makesavenoteenv{longtable}
\usepackage{graphicx}
\makeatletter
\def\maxwidth{\ifdim\Gin@nat@width>\linewidth\linewidth\else\Gin@nat@width\fi}
\def\maxheight{\ifdim\Gin@nat@height>\textheight\textheight\else\Gin@nat@height\fi}
\makeatother
% Scale images if necessary, so that they will not overflow the page
% margins by default, and it is still possible to overwrite the defaults
% using explicit options in \includegraphics[width, height, ...]{}
\setkeys{Gin}{width=\maxwidth,height=\maxheight,keepaspectratio}
% Set default figure placement to htbp
\makeatletter
\def\fps@figure{htbp}
\makeatother

\KOMAoption{captions}{tableheading}
\makeatletter
\makeatother
\makeatletter
\makeatother
\makeatletter
\@ifpackageloaded{caption}{}{\usepackage{caption}}
\AtBeginDocument{%
\ifdefined\contentsname
  \renewcommand*\contentsname{Table of contents}
\else
  \newcommand\contentsname{Table of contents}
\fi
\ifdefined\listfigurename
  \renewcommand*\listfigurename{List of Figures}
\else
  \newcommand\listfigurename{List of Figures}
\fi
\ifdefined\listtablename
  \renewcommand*\listtablename{List of Tables}
\else
  \newcommand\listtablename{List of Tables}
\fi
\ifdefined\figurename
  \renewcommand*\figurename{Figure}
\else
  \newcommand\figurename{Figure}
\fi
\ifdefined\tablename
  \renewcommand*\tablename{Table}
\else
  \newcommand\tablename{Table}
\fi
}
\@ifpackageloaded{float}{}{\usepackage{float}}
\floatstyle{ruled}
\@ifundefined{c@chapter}{\newfloat{codelisting}{h}{lop}}{\newfloat{codelisting}{h}{lop}[chapter]}
\floatname{codelisting}{Listing}
\newcommand*\listoflistings{\listof{codelisting}{List of Listings}}
\makeatother
\makeatletter
\@ifpackageloaded{caption}{}{\usepackage{caption}}
\@ifpackageloaded{subcaption}{}{\usepackage{subcaption}}
\makeatother
\makeatletter
\@ifpackageloaded{tcolorbox}{}{\usepackage[many]{tcolorbox}}
\makeatother
\makeatletter
\@ifundefined{shadecolor}{\definecolor{shadecolor}{rgb}{.97, .97, .97}}
\makeatother
\makeatletter
\makeatother
\ifLuaTeX
  \usepackage{selnolig}  % disable illegal ligatures
\fi
\IfFileExists{bookmark.sty}{\usepackage{bookmark}}{\usepackage{hyperref}}
\IfFileExists{xurl.sty}{\usepackage{xurl}}{} % add URL line breaks if available
\urlstyle{same} % disable monospaced font for URLs
\hypersetup{
  pdftitle={GSMST CS Club Cybersecurity Department Syllabus},
  pdfauthor={Anish Goyal; Bibek Bhattari; Garrett Rector; Henry Bui; Jonah Leopold; Parv Mahajan; Yubo Cao},
  colorlinks=true,
  linkcolor={blue},
  filecolor={Maroon},
  citecolor={Blue},
  urlcolor={Blue},
  pdfcreator={LaTeX via pandoc}}

\title{GSMST CS Club Cybersecurity Department Syllabus}
\author{Anish Goyal \and Bibek Bhattari \and Garrett Rector \and Henry
Bui \and Jonah Leopold \and Parv Mahajan \and Yubo Cao}
\date{}

\begin{document}
\maketitle
\ifdefined\Shaded\renewenvironment{Shaded}{\begin{tcolorbox}[sharp corners, boxrule=0pt, interior hidden, frame hidden, breakable, enhanced, borderline west={3pt}{0pt}{shadecolor}]}{\end{tcolorbox}}\fi

\renewcommand*\contentsname{Table of Contents}
{
\hypersetup{linkcolor=}
\setcounter{tocdepth}{6}
\tableofcontents
}
\newpage{}

\hypertarget{capture-the-flag}{%
\section{Capture the Flag}\label{capture-the-flag}}

\hypertarget{overview}{%
\subsection{Overview}\label{overview}}

Capture the Flag, or CTF, is largely concerned with offensive hacking.
Participating teams must first get a string with a certain format or
other content from the competition environment provided by the organizer
via ethical hacking techniques, and then submit it to the organizer to
receive points. Such information is referred to as ``flag.''

GSMST CS Club holds CTF workshops after school on Thursdays to prepare
for the annual picoCTF tournament in March. Out of the vast range of CTF
tasks, we will mostly focus on: general skills, reverse engineering,
binary exploitation, forensics, web exploitation, and cryptography.

\hypertarget{important-event-dates}{%
\subsection{Important Event Dates}\label{important-event-dates}}

\hypertarget{schoolwide-ctf-competition-0223202303092023}{%
\subsubsection{Schoolwide CTF Competition
(02/23/2023--03/09/2023)}\label{schoolwide-ctf-competition-0223202303092023}}

From 3:15 PM on Thursday, February 23 to 4:00 PM on Thursday, March 9,
GSMST CS Club will conduct a schoolwide CTF competition to prepare for
the picoCTF competition that begins on March 14.

\hypertarget{official-picoctf-competition-0314202303282023}{%
\subsubsection{Official picoCTF Competition
(03/14/2023--03/28/2023)}\label{official-picoctf-competition-0314202303282023}}

The official picoCTF competition will take place in the two weeks
between March 14 and March 18. The competition will be hosted on the
picoCTF website and will be open to all students in the United States.
All GSMST CS Club members are encouraged to compete and put your CTF
skills to the test!

\hypertarget{seminar-dates-and-topics}{%
\subsection{Seminar Dates and Topics}\label{seminar-dates-and-topics}}

\hypertarget{introduction-to-barcode-qr-code-scanning-1122023}{%
\subsubsection{Introduction to Barcode \& QR Code Scanning
(1/12/2023)}\label{introduction-to-barcode-qr-code-scanning-1122023}}

Barcodes and QR codes can be used in CTF challenges as a way to hide or
encode information that players need to find or decode in order to solve
the challenge. In this seminar, we will learn how barcode code scanners
work behind the scenes to decode barcodes and QR codes. We will also
learn how to use a QR code generator to create our own QR codes.

\newpage{}

\hypertarget{cyberpatriot}{%
\section{CyberPatriot}\label{cyberpatriot}}

\hypertarget{overview-1}{%
\subsection{Overview}\label{overview-1}}

CyberPatriot is a national defensive cybersecurity competition that
takes place during the majority of the school year. The competition
focuses on locating and fixing security vulnerabilities located within
virtual machines and network topologies. GSMST CS Club hosts
CyberPatriot seminars after school every Friday to prepare for the
competition and teach the best practices and techniques for securing
operating systems.

\hypertarget{seminar-dates-and-topics-1}{%
\subsection{Seminar Dates and Topics}\label{seminar-dates-and-topics-1}}

\hypertarget{semifinals-preparation-11220231202023}{%
\subsubsection{Semifinals Preparation
(1/12/2023--1/20/2023)}\label{semifinals-preparation-11220231202023}}

Because CyberPatriot semifinals will take place on Saturday, January 21,
we will dedicate our CyberPatriot seminars leading up to the competition
to prepare (1/13 \& 1/20). We will also be conducting two Cisco Packet
Tracer seminars on the Thursdays before semifinals to prepare teams for
the Cisco Packet Tracer challenge (1/12 \& 1/19).

\hypertarget{introduction-to-cisco-packet-tracer-1122023}{%
\paragraph{Introduction to Cisco Packet Tracer
(1/12/2023)}\label{introduction-to-cisco-packet-tracer-1122023}}

In this seminar, you will learn how to use Cisco Packet Tracer to
construct a network topology and configure devices to interact with one
anothere. Routers, switches, and hubs will be discussed as basic network
components. In addition, you will learn how to configure SSH, VLAN, and
interfaces. The following CLI modes will be covered: user execution
mode, privileged mode, global configuration mode, interface
configuration mode, and ROMMON mode.

\hypertarget{securing-harden-ubuntu-fedora}{%
\paragraph{Securing \& Hardening Ubuntu/Fedora
(1/13/2023)}\label{securing-harden-ubuntu-fedora}}

This session will teach you how to protect and harden Ubuntu and Fedora
by using access control lists (ACLs), discretionary access control
(DAC), and mandatory access control (MAC). The distinction between
Redhat-family Linux and Debain-family Linux is also discussed, as well
as their similarities and distinctions. For Redhat-family linux, an
overview of the use of `dnf,' `yum,' and `rpm' is also provided.
Capabilities such as SELinux, Apparmor, `getfattr,' `getacl,' and
`getfcap' are also addressed.

\newpage{}

\hypertarget{advanced-cisco-packet-tracer-1192023}{%
\paragraph{Advanced Cisco Packet Tracer
(1/19/2023)}\label{advanced-cisco-packet-tracer-1192023}}

a

\hypertarget{securing-hardening-windowswindows-server-1202023}{%
\paragraph{Securing \& Hardening Windows/Windows Server
(1/20/2023)}\label{securing-hardening-windowswindows-server-1202023}}

a

\hypertarget{content-seminars-12720234282023}{%
\subsubsection{Content Seminars
(1/27/2023--4/28/2023)}\label{content-seminars-12720234282023}}

\hypertarget{introduction-to-scripting-1272023}{%
\paragraph{Introduction to Scripting
(1/27/2023)}\label{introduction-to-scripting-1272023}}

In this seminar, you will learn basics of 3 scripting languages: - Bash
(Linux) - Python (Platform-agnostic) - Powershell (Windows)

As you embark on this seminar, it's recommended that you have prior
experience with AP CSP or AP CSA, though it is not a requirement. The
primary demonstration tool that will be used is Bash on an Ubuntu 22.04
VM. However, the knowledge gained during this seminar is considered to
be generally applicable.

We will cover topics such as output redirection, standard input/output,
environmental variables, and profile files like \texttt{.bashrc} or
\texttt{\textasciitilde{}/Documents/Powershell/Microsoft.PowerShell\_profile.ps1}.
The goal of this seminar is to empower you with the ability to read and
understand code and reference manuals, rather than feeling overwhelmed
and unsure.

Please note that this seminar will not be a comprehensive overview of
all CLI tools, as it is assumed that you have a basic understanding of
them. Instead, the focus will be on language structure and extremely
common operations.

\begin{itemize}
\item
  if/else

\begin{Shaded}
\begin{Highlighting}[]
\ControlFlowTok{if} \BuiltInTok{[} \VariableTok{$1} \OtherTok{{-}gt}\NormalTok{ 0 }\BuiltInTok{]}\KeywordTok{;} \ControlFlowTok{then}
  \BuiltInTok{echo} \StringTok{"Positive"}
\ControlFlowTok{fi}
\end{Highlighting}
\end{Shaded}

\begin{Shaded}
\begin{Highlighting}[]
\ControlFlowTok{if}\NormalTok{ sys.argv[}\DecValTok{1}\NormalTok{] }\OperatorTok{\textgreater{}} \DecValTok{0}\NormalTok{:}
  \BuiltInTok{print}\NormalTok{(}\StringTok{"Positive"}\NormalTok{)}
\end{Highlighting}
\end{Shaded}

\begin{Shaded}
\begin{Highlighting}[]
\KeywordTok{if} \OperatorTok{(}\VariableTok{$args}\OperatorTok{[}\NormalTok{0}\OperatorTok{]} \OperatorTok{{-}}\NormalTok{gt 0}\OperatorTok{)} \OperatorTok{\{}
  \FunctionTok{Write{-}Output} \StringTok{"Positive"}
\OperatorTok{\}}
\end{Highlighting}
\end{Shaded}
\item
  for

\begin{Shaded}
\begin{Highlighting}[]
\ControlFlowTok{for}\NormalTok{ i }\KeywordTok{in} \DataTypeTok{\{}\DecValTok{1}\DataTypeTok{..}\DecValTok{10}\DataTypeTok{\}}\KeywordTok{;} \ControlFlowTok{do}
  \BuiltInTok{echo} \VariableTok{$i}
\ControlFlowTok{done}
\end{Highlighting}
\end{Shaded}

\begin{Shaded}
\begin{Highlighting}[]
\ControlFlowTok{for}\NormalTok{ i }\KeywordTok{in} \BuiltInTok{range}\NormalTok{(}\DecValTok{1}\NormalTok{, }\DecValTok{11}\NormalTok{):}
  \BuiltInTok{print}\NormalTok{(i)}
\end{Highlighting}
\end{Shaded}

\begin{Shaded}
\begin{Highlighting}[]
\KeywordTok{for} \OperatorTok{(}\VariableTok{$i} \OperatorTok{=}\NormalTok{ 1}\OperatorTok{;} \VariableTok{$i} \OperatorTok{{-}}\NormalTok{le 10}\OperatorTok{;} \VariableTok{$i}\OperatorTok{++)} \OperatorTok{\{}
  \FunctionTok{Write{-}Output} \VariableTok{$i}
\OperatorTok{\}}
\end{Highlighting}
\end{Shaded}
\item
  while

\begin{Shaded}
\begin{Highlighting}[]
\VariableTok{i}\OperatorTok{=}\NormalTok{1}
\ControlFlowTok{while} \BuiltInTok{[} \VariableTok{$i} \OtherTok{{-}le}\NormalTok{ 10 }\BuiltInTok{]}\KeywordTok{;} \ControlFlowTok{do}
  \BuiltInTok{echo} \VariableTok{$i}
  \VariableTok{i}\OperatorTok{=}\VariableTok{$((i}\OperatorTok{+}\DecValTok{1}\VariableTok{))}
\ControlFlowTok{done}
\end{Highlighting}
\end{Shaded}

\begin{Shaded}
\begin{Highlighting}[]
\NormalTok{i }\OperatorTok{=} \DecValTok{1}
\ControlFlowTok{while}\NormalTok{ i }\OperatorTok{\textless{}=} \DecValTok{10}\NormalTok{:}
  \BuiltInTok{print}\NormalTok{(i)}
\NormalTok{  i }\OperatorTok{+=} \DecValTok{1}
\end{Highlighting}
\end{Shaded}

\begin{Shaded}
\begin{Highlighting}[]
\VariableTok{$i} \OperatorTok{=}\NormalTok{ 1}
\KeywordTok{while} \OperatorTok{(}\VariableTok{$i} \OperatorTok{{-}}\NormalTok{le 10}\OperatorTok{)} \OperatorTok{\{}
  \FunctionTok{Write{-}Output} \VariableTok{$i}
  \VariableTok{$i}\OperatorTok{++}
\OperatorTok{\}}
\end{Highlighting}
\end{Shaded}
\item
  switch

\begin{Shaded}
\begin{Highlighting}[]
\ControlFlowTok{case} \VariableTok{$1} \KeywordTok{in}
  \SpecialStringTok{1}\KeywordTok{)}
    \BuiltInTok{echo} \StringTok{"Anish YYDS"}
    \ControlFlowTok{;;}
  \PreprocessorTok{*}\KeywordTok{)}
    \BuiltInTok{echo} \StringTok{"Other"}
    \ControlFlowTok{;;}
\ControlFlowTok{esac}
\end{Highlighting}
\end{Shaded}

\begin{Shaded}
\begin{Highlighting}[]
\CommentTok{\# Compatibility with Python 3.9 and below:}
\NormalTok{switch }\OperatorTok{=}\NormalTok{ \{}
  \DecValTok{1}\NormalTok{: }\KeywordTok{lambda}\NormalTok{ : }\BuiltInTok{print}\NormalTok{(}\StringTok{"Anish YYDS"}\NormalTok{),}
  \StringTok{"\_"}\NormalTok{: }\KeywordTok{lambda}\NormalTok{ : }\BuiltInTok{print}\NormalTok{(}\StringTok{"Other"}\NormalTok{)}
\NormalTok{\}}
\NormalTok{switch.get(sys.argv[}\DecValTok{1}\NormalTok{], switch[}\StringTok{"\_"}\NormalTok{])()}

\CommentTok{\# Python 3.10+:}
\ControlFlowTok{match}\NormalTok{ sys.argv[}\DecValTok{1}\NormalTok{]:}
  \ControlFlowTok{case} \DecValTok{1}\NormalTok{:}
    \BuiltInTok{print}\NormalTok{(}\StringTok{"Anish YYDS"}\NormalTok{)}
  \ControlFlowTok{case}\NormalTok{ \_:}
    \BuiltInTok{print}\NormalTok{(}\StringTok{"Other"}\NormalTok{)}
\end{Highlighting}
\end{Shaded}

\begin{Shaded}
\begin{Highlighting}[]
\KeywordTok{switch} \OperatorTok{(}\VariableTok{$args}\OperatorTok{[}\NormalTok{0}\OperatorTok{])} \OperatorTok{\{}
\NormalTok{  1 }\OperatorTok{\{} \FunctionTok{Write{-}Output} \StringTok{"Anish YYDS"} \OperatorTok{\}}
\NormalTok{  default }\OperatorTok{\{} \FunctionTok{Write{-}Output} \StringTok{"Other"} \OperatorTok{\}}
\OperatorTok{\}}
\end{Highlighting}
\end{Shaded}
\item
  functions

\begin{Shaded}
\begin{Highlighting}[]
\KeywordTok{function}\FunctionTok{ hello()} \KeywordTok{\{}
  \BuiltInTok{echo} \StringTok{"Hello, }\VariableTok{$1}\StringTok{!"}
\KeywordTok{\}}
\ExtensionTok{hello} \StringTok{"World"}
\end{Highlighting}
\end{Shaded}

\begin{Shaded}
\begin{Highlighting}[]
\KeywordTok{def}\NormalTok{ hello(name):}
  \BuiltInTok{print}\NormalTok{(}\SpecialStringTok{f"Hello, }\SpecialCharTok{\{}\NormalTok{name}\SpecialCharTok{\}}\SpecialStringTok{!"}\NormalTok{)}
\NormalTok{hello(}\StringTok{"World"}\NormalTok{)}
\end{Highlighting}
\end{Shaded}

\begin{Shaded}
\begin{Highlighting}[]
\KeywordTok{function}\NormalTok{ Write}\OperatorTok{{-}}\NormalTok{Hello}\OperatorTok{(}\VariableTok{$name}\OperatorTok{)} \OperatorTok{\{}
  \FunctionTok{Write{-}Output} \StringTok{"Hello, $name!"}
\OperatorTok{\}}
\NormalTok{Write}\OperatorTok{{-}}\NormalTok{Hello }\OperatorTok{{-}}\NormalTok{name }\StringTok{"World"}
\end{Highlighting}
\end{Shaded}
\item
  Mapping data structures

\begin{Shaded}
\begin{Highlighting}[]
\BuiltInTok{declare} \AttributeTok{{-}A} \VariableTok{dict}
\VariableTok{dict}\OperatorTok{[}\StringTok{"key"}\OperatorTok{]=}\StringTok{"value"}
\BuiltInTok{echo} \VariableTok{$\{dict}\OperatorTok{[}\StringTok{"key"}\OperatorTok{]}\VariableTok{\}}
\end{Highlighting}
\end{Shaded}

\begin{Shaded}
\begin{Highlighting}[]
\BuiltInTok{dict} \OperatorTok{=}\NormalTok{ \{}\StringTok{"key"}\NormalTok{: }\StringTok{"value"}\NormalTok{\}}
\BuiltInTok{print}\NormalTok{(}\BuiltInTok{dict}\NormalTok{[}\StringTok{"key"}\NormalTok{])}
\end{Highlighting}
\end{Shaded}

\begin{Shaded}
\begin{Highlighting}[]
\VariableTok{$dict} \OperatorTok{=}\NormalTok{ @}\OperatorTok{\{}\NormalTok{key}\OperatorTok{=}\StringTok{"value"}\OperatorTok{\}}
\FunctionTok{Write{-}Output} \VariableTok{$dict}\OperatorTok{.}\FunctionTok{key}
\end{Highlighting}
\end{Shaded}
\item
  List data structures

\begin{Shaded}
\begin{Highlighting}[]
\VariableTok{list}\OperatorTok{=}\VariableTok{(}\NormalTok{1 2 3}\VariableTok{)}
\BuiltInTok{echo} \VariableTok{$\{list}\OperatorTok{[}\DecValTok{0}\OperatorTok{]}\VariableTok{\}}
\end{Highlighting}
\end{Shaded}

\begin{Shaded}
\begin{Highlighting}[]
\BuiltInTok{list} \OperatorTok{=}\NormalTok{ [}\DecValTok{1}\NormalTok{, }\DecValTok{2}\NormalTok{, }\DecValTok{3}\NormalTok{]}
\BuiltInTok{print}\NormalTok{(}\BuiltInTok{list}\NormalTok{[}\DecValTok{0}\NormalTok{])}
\end{Highlighting}
\end{Shaded}

\begin{Shaded}
\begin{Highlighting}[]
\VariableTok{$list} \OperatorTok{=}\NormalTok{ 1}\OperatorTok{,}\NormalTok{ 2}\OperatorTok{,}\NormalTok{ 3}
\FunctionTok{Write{-}Output} \VariableTok{$list}\OperatorTok{[}\NormalTok{0}\OperatorTok{]}
\end{Highlighting}
\end{Shaded}
\end{itemize}

Only extremely common operations will be introduced:

\begin{itemize}
\tightlist
\item
  \texttt{ls}, \texttt{Get-ChildItem}, and \texttt{Path.iterdir()}
\item
  \texttt{cd}, \texttt{Set-Location}, and \texttt{os.chdir()}
\item
  \texttt{mkdir}, \texttt{New-Item}, and \texttt{Path.mkdir()}
\item
  \texttt{rm}, \texttt{Remove-Item}, and \texttt{Path.unlink()}
\item
  \texttt{cat}, \texttt{Get-Content}, and \texttt{Path.read\_text()}
\item
  \texttt{echo}, \texttt{Write-Output}, and \texttt{print()}
\item
  \texttt{grep}, \texttt{Select-String}, and \texttt{re.search()}
\item
  \texttt{sed}, \texttt{Select-String}, and \texttt{re.sub()}
\item
  \texttt{awk}, \texttt{Select-String}, and \texttt{re.split()}
\item
  \texttt{wc}, \texttt{Measure-Object}, and \texttt{len()}
\end{itemize}

\hypertarget{acl-access-control-list-2102023}{%
\paragraph{ACL (access control list)
(2/10/2023)}\label{acl-access-control-list-2102023}}

In this seminar, you will gain a comprehensive understanding of the ACL
(access control list) feature in Windows and how to effectively use it
to secure your system. One of the key tools you will learn about is
active directory, which allows for systematic management of ACLs, user
accounts, and group memberships across multiple systems with a single
source of truth.

You will also learn about the principle of least privilege and its
importance in securing your system. Topics such as DCom, account policy,
user policy, and group policy will be covered in the user configuration
section. As an example, Autoplay will be disabled. Additionally, you
will learn about password complexity requirements and how to set local
and domain password policies. As an optional topic, you may also learn
about how to set Cisco password policies. ACLs in Linux is already
covered at @securing-harden-ubuntu-fedora and shall not be covered in
this seminar.

By the end of this seminar, you will be equipped with the knowledge and
skills needed to effectively secure your system using ACLs and other
related tools and techniques.

\hypertarget{common-web-services-2242023}{%
\paragraph{Common Web Services
(2/24/2023)}\label{common-web-services-2242023}}

\hypertarget{section}{%
\paragraph{(3/3/2023)}\label{section}}

\hypertarget{section-1}{%
\paragraph{(3/10/2023)}\label{section-1}}

\hypertarget{section-2}{%
\paragraph{(3/24/2023)}\label{section-2}}

\hypertarget{section-3}{%
\paragraph{(3/31/2023)}\label{section-3}}

Annouce CS Club officer application

\hypertarget{section-4}{%
\paragraph{(4/14/2023)}\label{section-4}}

\hypertarget{section-5}{%
\paragraph{(4/21/2023)}\label{section-5}}

\hypertarget{section-6}{%
\paragraph{(4/28/2023)}\label{section-6}}

End of year celebration



\end{document}
