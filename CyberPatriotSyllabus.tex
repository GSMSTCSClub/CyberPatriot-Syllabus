% Options for packages loaded elsewhere
\PassOptionsToPackage{unicode}{hyperref}
\PassOptionsToPackage{hyphens}{url}
\PassOptionsToPackage{dvipsnames,svgnames,x11names}{xcolor}
%
\documentclass[
  letterpaper,
  DIV=11,
  numbers=noendperiod]{scrartcl}

\usepackage{amsmath,amssymb}
\usepackage{lmodern}
\usepackage{iftex}
\ifPDFTeX
  \usepackage[T1]{fontenc}
  \usepackage[utf8]{inputenc}
  \usepackage{textcomp} % provide euro and other symbols
\else % if luatex or xetex
  \usepackage{unicode-math}
  \defaultfontfeatures{Scale=MatchLowercase}
  \defaultfontfeatures[\rmfamily]{Ligatures=TeX,Scale=1}
\fi
% Use upquote if available, for straight quotes in verbatim environments
\IfFileExists{upquote.sty}{\usepackage{upquote}}{}
\IfFileExists{microtype.sty}{% use microtype if available
  \usepackage[]{microtype}
  \UseMicrotypeSet[protrusion]{basicmath} % disable protrusion for tt fonts
}{}
\makeatletter
\@ifundefined{KOMAClassName}{% if non-KOMA class
  \IfFileExists{parskip.sty}{%
    \usepackage{parskip}
  }{% else
    \setlength{\parindent}{0pt}
    \setlength{\parskip}{6pt plus 2pt minus 1pt}}
}{% if KOMA class
  \KOMAoptions{parskip=half}}
\makeatother
\usepackage{xcolor}
\setlength{\emergencystretch}{3em} % prevent overfull lines
\setcounter{secnumdepth}{5}
% Make \paragraph and \subparagraph free-standing
\ifx\paragraph\undefined\else
  \let\oldparagraph\paragraph
  \renewcommand{\paragraph}[1]{\oldparagraph{#1}\mbox{}}
\fi
\ifx\subparagraph\undefined\else
  \let\oldsubparagraph\subparagraph
  \renewcommand{\subparagraph}[1]{\oldsubparagraph{#1}\mbox{}}
\fi

\usepackage{color}
\usepackage{fancyvrb}
\newcommand{\VerbBar}{|}
\newcommand{\VERB}{\Verb[commandchars=\\\{\}]}
\DefineVerbatimEnvironment{Highlighting}{Verbatim}{commandchars=\\\{\}}
% Add ',fontsize=\small' for more characters per line
\usepackage{framed}
\definecolor{shadecolor}{RGB}{241,243,245}
\newenvironment{Shaded}{\begin{snugshade}}{\end{snugshade}}
\newcommand{\AlertTok}[1]{\textcolor[rgb]{0.68,0.00,0.00}{#1}}
\newcommand{\AnnotationTok}[1]{\textcolor[rgb]{0.37,0.37,0.37}{#1}}
\newcommand{\AttributeTok}[1]{\textcolor[rgb]{0.40,0.45,0.13}{#1}}
\newcommand{\BaseNTok}[1]{\textcolor[rgb]{0.68,0.00,0.00}{#1}}
\newcommand{\BuiltInTok}[1]{\textcolor[rgb]{0.00,0.23,0.31}{#1}}
\newcommand{\CharTok}[1]{\textcolor[rgb]{0.13,0.47,0.30}{#1}}
\newcommand{\CommentTok}[1]{\textcolor[rgb]{0.37,0.37,0.37}{#1}}
\newcommand{\CommentVarTok}[1]{\textcolor[rgb]{0.37,0.37,0.37}{\textit{#1}}}
\newcommand{\ConstantTok}[1]{\textcolor[rgb]{0.56,0.35,0.01}{#1}}
\newcommand{\ControlFlowTok}[1]{\textcolor[rgb]{0.00,0.23,0.31}{#1}}
\newcommand{\DataTypeTok}[1]{\textcolor[rgb]{0.68,0.00,0.00}{#1}}
\newcommand{\DecValTok}[1]{\textcolor[rgb]{0.68,0.00,0.00}{#1}}
\newcommand{\DocumentationTok}[1]{\textcolor[rgb]{0.37,0.37,0.37}{\textit{#1}}}
\newcommand{\ErrorTok}[1]{\textcolor[rgb]{0.68,0.00,0.00}{#1}}
\newcommand{\ExtensionTok}[1]{\textcolor[rgb]{0.00,0.23,0.31}{#1}}
\newcommand{\FloatTok}[1]{\textcolor[rgb]{0.68,0.00,0.00}{#1}}
\newcommand{\FunctionTok}[1]{\textcolor[rgb]{0.28,0.35,0.67}{#1}}
\newcommand{\ImportTok}[1]{\textcolor[rgb]{0.00,0.46,0.62}{#1}}
\newcommand{\InformationTok}[1]{\textcolor[rgb]{0.37,0.37,0.37}{#1}}
\newcommand{\KeywordTok}[1]{\textcolor[rgb]{0.00,0.23,0.31}{#1}}
\newcommand{\NormalTok}[1]{\textcolor[rgb]{0.00,0.23,0.31}{#1}}
\newcommand{\OperatorTok}[1]{\textcolor[rgb]{0.37,0.37,0.37}{#1}}
\newcommand{\OtherTok}[1]{\textcolor[rgb]{0.00,0.23,0.31}{#1}}
\newcommand{\PreprocessorTok}[1]{\textcolor[rgb]{0.68,0.00,0.00}{#1}}
\newcommand{\RegionMarkerTok}[1]{\textcolor[rgb]{0.00,0.23,0.31}{#1}}
\newcommand{\SpecialCharTok}[1]{\textcolor[rgb]{0.37,0.37,0.37}{#1}}
\newcommand{\SpecialStringTok}[1]{\textcolor[rgb]{0.13,0.47,0.30}{#1}}
\newcommand{\StringTok}[1]{\textcolor[rgb]{0.13,0.47,0.30}{#1}}
\newcommand{\VariableTok}[1]{\textcolor[rgb]{0.07,0.07,0.07}{#1}}
\newcommand{\VerbatimStringTok}[1]{\textcolor[rgb]{0.13,0.47,0.30}{#1}}
\newcommand{\WarningTok}[1]{\textcolor[rgb]{0.37,0.37,0.37}{\textit{#1}}}

\providecommand{\tightlist}{%
  \setlength{\itemsep}{0pt}\setlength{\parskip}{0pt}}\usepackage{longtable,booktabs,array}
\usepackage{calc} % for calculating minipage widths
% Correct order of tables after \paragraph or \subparagraph
\usepackage{etoolbox}
\makeatletter
\patchcmd\longtable{\par}{\if@noskipsec\mbox{}\fi\par}{}{}
\makeatother
% Allow footnotes in longtable head/foot
\IfFileExists{footnotehyper.sty}{\usepackage{footnotehyper}}{\usepackage{footnote}}
\makesavenoteenv{longtable}
\usepackage{graphicx}
\makeatletter
\def\maxwidth{\ifdim\Gin@nat@width>\linewidth\linewidth\else\Gin@nat@width\fi}
\def\maxheight{\ifdim\Gin@nat@height>\textheight\textheight\else\Gin@nat@height\fi}
\makeatother
% Scale images if necessary, so that they will not overflow the page
% margins by default, and it is still possible to overwrite the defaults
% using explicit options in \includegraphics[width, height, ...]{}
\setkeys{Gin}{width=\maxwidth,height=\maxheight,keepaspectratio}
% Set default figure placement to htbp
\makeatletter
\def\fps@figure{htbp}
\makeatother

\KOMAoption{captions}{tableheading}
\makeatletter
\makeatother
\makeatletter
\makeatother
\makeatletter
\@ifpackageloaded{caption}{}{\usepackage{caption}}
\AtBeginDocument{%
\ifdefined\contentsname
  \renewcommand*\contentsname{Table of contents}
\else
  \newcommand\contentsname{Table of contents}
\fi
\ifdefined\listfigurename
  \renewcommand*\listfigurename{List of Figures}
\else
  \newcommand\listfigurename{List of Figures}
\fi
\ifdefined\listtablename
  \renewcommand*\listtablename{List of Tables}
\else
  \newcommand\listtablename{List of Tables}
\fi
\ifdefined\figurename
  \renewcommand*\figurename{Figure}
\else
  \newcommand\figurename{Figure}
\fi
\ifdefined\tablename
  \renewcommand*\tablename{Table}
\else
  \newcommand\tablename{Table}
\fi
}
\@ifpackageloaded{float}{}{\usepackage{float}}
\floatstyle{ruled}
\@ifundefined{c@chapter}{\newfloat{codelisting}{h}{lop}}{\newfloat{codelisting}{h}{lop}[chapter]}
\floatname{codelisting}{Listing}
\newcommand*\listoflistings{\listof{codelisting}{List of Listings}}
\makeatother
\makeatletter
\@ifpackageloaded{caption}{}{\usepackage{caption}}
\@ifpackageloaded{subcaption}{}{\usepackage{subcaption}}
\makeatother
\makeatletter
\@ifpackageloaded{tcolorbox}{}{\usepackage[many]{tcolorbox}}
\makeatother
\makeatletter
\@ifundefined{shadecolor}{\definecolor{shadecolor}{rgb}{.97, .97, .97}}
\makeatother
\makeatletter
\makeatother
\ifLuaTeX
  \usepackage{selnolig}  % disable illegal ligatures
\fi
\IfFileExists{bookmark.sty}{\usepackage{bookmark}}{\usepackage{hyperref}}
\IfFileExists{xurl.sty}{\usepackage{xurl}}{} % add URL line breaks if available
\urlstyle{same} % disable monospaced font for URLs
\hypersetup{
  pdftitle={GSMST CS Club Cybersecurity Department Syllabus},
  pdfauthor={Anish Goyal; Bibek Bhattari; Garrett Rector; Henry Bui; Jonah Leopold; Parv Mahajan; Yubo Cao},
  colorlinks=true,
  linkcolor={blue},
  filecolor={Maroon},
  citecolor={Blue},
  urlcolor={Blue},
  pdfcreator={LaTeX via pandoc}}

\title{GSMST CS Club Cybersecurity Department Syllabus}
\author{Anish Goyal \and Bibek Bhattari \and Garrett Rector \and Henry
Bui \and Jonah Leopold \and Parv Mahajan \and Yubo Cao}
\date{}

\begin{document}
\maketitle
\ifdefined\Shaded\renewenvironment{Shaded}{\begin{tcolorbox}[interior hidden, boxrule=0pt, borderline west={3pt}{0pt}{shadecolor}, breakable, enhanced, frame hidden, sharp corners]}{\end{tcolorbox}}\fi

\renewcommand*\contentsname{Table of Contents}
{
\hypersetup{linkcolor=}
\setcounter{tocdepth}{6}
\tableofcontents
}
\newpage{}

\hypertarget{capture-the-flag}{%
\section{Capture the Flag}\label{capture-the-flag}}

\hypertarget{overview}{%
\subsection{Overview}\label{overview}}

Capture the Flag, or CTF, is largely concerned with offensive hacking.
Participating teams must first get a string with a certain format or
other content from the competition environment provided by the organizer
via ethical hacking techniques, and then submit it to the organizer to
receive points. Such information is referred to as ``flag.''

GSMST CS Club holds CTF workshops after school on Thursdays to prepare
for the annual picoCTF tournament in March. Out of the vast range of CTF
tasks, we will mostly focus on: general skills, reverse engineering,
binary exploitation, forensics, web exploitation, and cryptography.

\hypertarget{important-event-dates}{%
\subsection{Important Event Dates}\label{important-event-dates}}

\hypertarget{schoolwide-ctf-competition-0223202303092023}{%
\subsubsection{Schoolwide CTF Competition
(02/23/2023--03/09/2023)}\label{schoolwide-ctf-competition-0223202303092023}}

From 3:15 PM on Thursday, February 23 to 4:00 PM on Thursday, March 9,
GSMST CS Club will conduct a schoolwide CTF competition to prepare for
the picoCTF competition that begins on March 14.

\hypertarget{official-picoctf-competition-0314202303282023}{%
\subsubsection{Official picoCTF Competition
(03/14/2023--03/28/2023)}\label{official-picoctf-competition-0314202303282023}}

The official picoCTF competition will take place in the two weeks
between March 14 and March 18. The competition will be hosted on the
picoCTF website and will be open to all students in the United States.
All GSMST CS Club members are encouraged to compete and put your CTF
skills to the test!

\hypertarget{seminar-dates-and-topics}{%
\subsection{Seminar Dates and Topics}\label{seminar-dates-and-topics}}

\hypertarget{introduction-to-barcode-qr-code-scanning-1122023}{%
\subsubsection{Introduction to Barcode \& QR Code Scanning
(1/12/2023)}\label{introduction-to-barcode-qr-code-scanning-1122023}}

Barcodes and QR codes can be used in CTF challenges as a way to hide or
encode information that players need to find or decode in order to solve
the challenge. In this seminar, we will learn how barcode code scanners
work behind the scenes to decode barcodes and QR codes. We will also
learn how to use a QR code generator to create our own QR codes.

\newpage{}

\hypertarget{cyberpatriot}{%
\section{CyberPatriot}\label{cyberpatriot}}

\hypertarget{overview-1}{%
\subsection{Overview}\label{overview-1}}

CyberPatriot is a national defensive cybersecurity competition that
takes place during the majority of the school year. The competition
focuses on locating and fixing security vulnerabilities located within
virtual machines and network topologies. GSMST CS Club hosts
CyberPatriot seminars after school every Friday to prepare for the
competition and teach the best practices and techniques for securing
operating systems.

\hypertarget{important-event-dates-1}{%
\subsection{Important Event Dates}\label{important-event-dates-1}}

\hypertarget{cyberpatriot-semifinals-1212023}{%
\subsubsection{CyberPatriot Semifinals
(1/21/2023)}\label{cyberpatriot-semifinals-1212023}}

The CyberPatriot semifinals will take place on Saturday, January 21. The
semifinals will be held in the LLLH from 9:00 AM to 3:00 PM. The
semifinals will be a 6-hour competition in which teams will be given a
virtual machine and a network topology to secure. The semifinals will be
hosted by the CyberPatriot organization and will be open to all teams in
the United States.

\hypertarget{schoolwide-cyberpatriot-competition-2252023}{%
\subsubsection{Schoolwide CyberPatriot Competition
(2/25/2023)}\label{schoolwide-cyberpatriot-competition-2252023}}

The first schoolwide CyberPatriot competition! The competition will be
held in the LLLH from 9:00 AM-4:00 PM. The competition will be a 6-hour
competition in which teams will be given a virtual machine and a network
topology to secure. The competition will be hosted by GSMST CS Club and
will be open to all GSMST students. Students outside of the actual
CyberPatriot teams are encouraged to participate and RSVP on the Google
Form. Teams may only consist of four members, and each team must have at
least one member who is not a member of a current CyberPatriot team.

\hypertarget{schoolwide-cyberpatriot-compeition-482023}{%
\subsubsection{Schoolwide CyberPatriot Compeition
(4/8/2023)}\label{schoolwide-cyberpatriot-compeition-482023}}

The second (and final) schoolwide CyberPatriot competition! The
competition will be held in the LLLH from 9:00 AM-4:00 PM. Students
outside of the actual CyberPatriot teams are encouraged to participate
and RSVP on the Google Form. Teams may only consist of four members, and
each team must have at least one member who is not a member of a current
CyberPatriot team.

\hypertarget{seminar-dates-and-topics-1}{%
\subsection{Seminar Dates and Topics}\label{seminar-dates-and-topics-1}}

\hypertarget{semifinals-preparation-11220231202023}{%
\subsubsection{Semifinals Preparation
(1/12/2023--1/20/2023)}\label{semifinals-preparation-11220231202023}}

Because CyberPatriot semifinals will take place on Saturday, January 21,
we will dedicate our CyberPatriot seminars leading up to the competition
to prepare (1/13 \& 1/20). We will also be conducting two Cisco Packet
Tracer seminars on the Thursdays before semifinals to prepare teams for
the Cisco Packet Tracer challenge (1/12 \& 1/19).

\hypertarget{sec-harden-ubuntu-fedora}{%
\paragraph{Securing \& Hardening Ubuntu/Fedora
(1/13/2023)}\label{sec-harden-ubuntu-fedora}}

This session will teach you how to protect and harden Ubuntu and Fedora
by using access control lists (ACLs), discretionary access control
(DAC), and mandatory access control (MAC). The distinction between
Redhat-family Linux and Debain-family Linux is also discussed, as well
as their similarities and distinctions. For Redhat-family linux, an
overview of the use of `dnf,' `yum,' and `rpm' is also provided.
Capabilities such as SELinux, Apparmor, `getfattr,' `getacl,' and
`getfcap' are also addressed.

\hypertarget{sec-intro-cisco-packet-tracer}{%
\paragraph{Introduction to Cisco Packet Tracer
(1/19/2023)}\label{sec-intro-cisco-packet-tracer}}

In this seminar, you will learn how to use Cisco Packet Tracer to
construct a network topology and configure devices to interact with one
anothere. Routers, switches, and hubs will be discussed as basic network
components. In addition, you will learn how to configure SSH, VLAN, and
interfaces. The following CLI modes will be covered: user execution
mode, privileged mode, global configuration mode, interface
configuration mode, and ROMMON mode.

\newpage{}

\hypertarget{securing-hardening-windows-1202023}{%
\paragraph{Securing \& Hardening Windows
(1/20/2023)}\label{securing-hardening-windows-1202023}}

This session will delve into the various methods and tools available to
protect against common virus types and implement effective anti-malware
strategies. We will also explore the built-in security features of
Windows, such as the task manager, program management, and startup
management. Additionally, we will cover the importance of regularly
patching software, using the command prompt (CMD) for advanced
troubleshooting, and implementing internet security best practices.
Furthermore, we will discuss the Microsoft management console and the
SysInternals suite of tools, which can provide detailed information and
control over the operating system

\hypertarget{content-seminars-12720234282023}{%
\subsubsection{Content Seminars
(1/27/2023--4/28/2023)}\label{content-seminars-12720234282023}}

\hypertarget{introduction-to-scripting-1272023}{%
\paragraph{Introduction to Scripting
(1/27/2023)}\label{introduction-to-scripting-1272023}}

In this seminar, you will learn basics of 3 scripting languages: - Bash
(Linux) - Python (Platform-agnostic) - Powershell (Windows)

As you embark on this seminar, it's recommended that you have prior
experience with AP CSP or AP CSA, though it is not a requirement. The
primary demonstration tool that will be used is Bash on an Ubuntu 22.04
VM. However, the knowledge gained during this seminar is considered to
be generally applicable.

We will cover topics such as output redirection, standard input/output,
environmental variables, and profile files like \texttt{.bashrc} or
\texttt{\textasciitilde{}/Documents/Powershell/Microsoft.PowerShell\_profile.ps1}.
The goal of this seminar is to empower you with the ability to read and
understand code and reference manuals, rather than feeling overwhelmed
and unsure.

Please note that this seminar will not be a comprehensive overview of
all CLI tools, as it is assumed that you have a basic understanding of
them. Instead, the focus will be on language structure and extremely
common operations.

\begin{itemize}
\item
  if/else

\begin{Shaded}
\begin{Highlighting}[]
\ControlFlowTok{if} \BuiltInTok{[} \VariableTok{$1} \OtherTok{{-}gt}\NormalTok{ 0 }\BuiltInTok{]}\KeywordTok{;} \ControlFlowTok{then}
  \BuiltInTok{echo} \StringTok{"Positive"}
\ControlFlowTok{fi}
\end{Highlighting}
\end{Shaded}

\begin{Shaded}
\begin{Highlighting}[]
\ControlFlowTok{if}\NormalTok{ sys.argv[}\DecValTok{1}\NormalTok{] }\OperatorTok{\textgreater{}} \DecValTok{0}\NormalTok{:}
  \BuiltInTok{print}\NormalTok{(}\StringTok{"Positive"}\NormalTok{)}
\end{Highlighting}
\end{Shaded}

\begin{Shaded}
\begin{Highlighting}[]
\KeywordTok{if} \OperatorTok{(}\VariableTok{$args}\OperatorTok{[}\NormalTok{0}\OperatorTok{]} \OperatorTok{{-}}\NormalTok{gt 0}\OperatorTok{)} \OperatorTok{\{}
  \FunctionTok{Write{-}Output} \StringTok{"Positive"}
\OperatorTok{\}}
\end{Highlighting}
\end{Shaded}
\item
  for

\begin{Shaded}
\begin{Highlighting}[]
\ControlFlowTok{for}\NormalTok{ i }\KeywordTok{in} \DataTypeTok{\{}\DecValTok{1}\DataTypeTok{..}\DecValTok{10}\DataTypeTok{\}}\KeywordTok{;} \ControlFlowTok{do}
  \BuiltInTok{echo} \VariableTok{$i}
\ControlFlowTok{done}
\end{Highlighting}
\end{Shaded}

\begin{Shaded}
\begin{Highlighting}[]
\ControlFlowTok{for}\NormalTok{ i }\KeywordTok{in} \BuiltInTok{range}\NormalTok{(}\DecValTok{1}\NormalTok{, }\DecValTok{11}\NormalTok{):}
  \BuiltInTok{print}\NormalTok{(i)}
\end{Highlighting}
\end{Shaded}

\begin{Shaded}
\begin{Highlighting}[]
\KeywordTok{for} \OperatorTok{(}\VariableTok{$i} \OperatorTok{=}\NormalTok{ 1}\OperatorTok{;} \VariableTok{$i} \OperatorTok{{-}}\NormalTok{le 10}\OperatorTok{;} \VariableTok{$i}\OperatorTok{++)} \OperatorTok{\{}
  \FunctionTok{Write{-}Output} \VariableTok{$i}
\OperatorTok{\}}
\end{Highlighting}
\end{Shaded}
\item
  while

\begin{Shaded}
\begin{Highlighting}[]
\VariableTok{i}\OperatorTok{=}\NormalTok{1}
\ControlFlowTok{while} \BuiltInTok{[} \VariableTok{$i} \OtherTok{{-}le}\NormalTok{ 10 }\BuiltInTok{]}\KeywordTok{;} \ControlFlowTok{do}
  \BuiltInTok{echo} \VariableTok{$i}
  \VariableTok{i}\OperatorTok{=}\VariableTok{$((i}\OperatorTok{+}\DecValTok{1}\VariableTok{))}
\ControlFlowTok{done}
\end{Highlighting}
\end{Shaded}

\begin{Shaded}
\begin{Highlighting}[]
\NormalTok{i }\OperatorTok{=} \DecValTok{1}
\ControlFlowTok{while}\NormalTok{ i }\OperatorTok{\textless{}=} \DecValTok{10}\NormalTok{:}
  \BuiltInTok{print}\NormalTok{(i)}
\NormalTok{  i }\OperatorTok{+=} \DecValTok{1}
\end{Highlighting}
\end{Shaded}

\begin{Shaded}
\begin{Highlighting}[]
\VariableTok{$i} \OperatorTok{=}\NormalTok{ 1}
\KeywordTok{while} \OperatorTok{(}\VariableTok{$i} \OperatorTok{{-}}\NormalTok{le 10}\OperatorTok{)} \OperatorTok{\{}
  \FunctionTok{Write{-}Output} \VariableTok{$i}
  \VariableTok{$i}\OperatorTok{++}
\OperatorTok{\}}
\end{Highlighting}
\end{Shaded}
\item
  switch

\begin{Shaded}
\begin{Highlighting}[]
\ControlFlowTok{case} \VariableTok{$1} \KeywordTok{in}
  \SpecialStringTok{1}\KeywordTok{)}
    \BuiltInTok{echo} \StringTok{"Anish YYDS"}
    \ControlFlowTok{;;}
  \PreprocessorTok{*}\KeywordTok{)}
    \BuiltInTok{echo} \StringTok{"Other"}
    \ControlFlowTok{;;}
\ControlFlowTok{esac}
\end{Highlighting}
\end{Shaded}

\begin{Shaded}
\begin{Highlighting}[]
\CommentTok{\# Compatibility with Python 3.9 and below:}
\NormalTok{switch }\OperatorTok{=}\NormalTok{ \{}
  \DecValTok{1}\NormalTok{: }\KeywordTok{lambda}\NormalTok{ : }\BuiltInTok{print}\NormalTok{(}\StringTok{"Anish YYDS"}\NormalTok{),}
  \StringTok{"\_"}\NormalTok{: }\KeywordTok{lambda}\NormalTok{ : }\BuiltInTok{print}\NormalTok{(}\StringTok{"Other"}\NormalTok{)}
\NormalTok{\}}
\NormalTok{switch.get(sys.argv[}\DecValTok{1}\NormalTok{], switch[}\StringTok{"\_"}\NormalTok{])()}

\CommentTok{\# Python 3.10+:}
\ControlFlowTok{match}\NormalTok{ sys.argv[}\DecValTok{1}\NormalTok{]:}
  \ControlFlowTok{case} \DecValTok{1}\NormalTok{:}
    \BuiltInTok{print}\NormalTok{(}\StringTok{"Anish YYDS"}\NormalTok{)}
  \ControlFlowTok{case}\NormalTok{ \_:}
    \BuiltInTok{print}\NormalTok{(}\StringTok{"Other"}\NormalTok{)}
\end{Highlighting}
\end{Shaded}

\begin{Shaded}
\begin{Highlighting}[]
\KeywordTok{switch} \OperatorTok{(}\VariableTok{$args}\OperatorTok{[}\NormalTok{0}\OperatorTok{])} \OperatorTok{\{}
\NormalTok{  1 }\OperatorTok{\{} \FunctionTok{Write{-}Output} \StringTok{"Anish YYDS"} \OperatorTok{\}}
\NormalTok{  default }\OperatorTok{\{} \FunctionTok{Write{-}Output} \StringTok{"Other"} \OperatorTok{\}}
\OperatorTok{\}}
\end{Highlighting}
\end{Shaded}
\item
  functions

\begin{Shaded}
\begin{Highlighting}[]
\KeywordTok{function}\FunctionTok{ hello()} \KeywordTok{\{}
  \BuiltInTok{echo} \StringTok{"Hello, }\VariableTok{$1}\StringTok{!"}
\KeywordTok{\}}
\ExtensionTok{hello} \StringTok{"World"}
\end{Highlighting}
\end{Shaded}

\begin{Shaded}
\begin{Highlighting}[]
\KeywordTok{def}\NormalTok{ hello(name):}
  \BuiltInTok{print}\NormalTok{(}\SpecialStringTok{f"Hello, }\SpecialCharTok{\{}\NormalTok{name}\SpecialCharTok{\}}\SpecialStringTok{!"}\NormalTok{)}
\NormalTok{hello(}\StringTok{"World"}\NormalTok{)}
\end{Highlighting}
\end{Shaded}

\begin{Shaded}
\begin{Highlighting}[]
\KeywordTok{function}\NormalTok{ Write}\OperatorTok{{-}}\NormalTok{Hello}\OperatorTok{(}\VariableTok{$name}\OperatorTok{)} \OperatorTok{\{}
  \FunctionTok{Write{-}Output} \StringTok{"Hello, $name!"}
\OperatorTok{\}}
\NormalTok{Write}\OperatorTok{{-}}\NormalTok{Hello }\OperatorTok{{-}}\NormalTok{name }\StringTok{"World"}
\end{Highlighting}
\end{Shaded}
\item
  Mapping data structures

\begin{Shaded}
\begin{Highlighting}[]
\BuiltInTok{declare} \AttributeTok{{-}A} \VariableTok{dict}
\VariableTok{dict}\OperatorTok{[}\StringTok{"key"}\OperatorTok{]=}\StringTok{"value"}
\BuiltInTok{echo} \VariableTok{$\{dict}\OperatorTok{[}\StringTok{"key"}\OperatorTok{]}\VariableTok{\}}
\end{Highlighting}
\end{Shaded}

\begin{Shaded}
\begin{Highlighting}[]
\BuiltInTok{dict} \OperatorTok{=}\NormalTok{ \{}\StringTok{"key"}\NormalTok{: }\StringTok{"value"}\NormalTok{\}}
\BuiltInTok{print}\NormalTok{(}\BuiltInTok{dict}\NormalTok{[}\StringTok{"key"}\NormalTok{])}
\end{Highlighting}
\end{Shaded}

\begin{Shaded}
\begin{Highlighting}[]
\VariableTok{$dict} \OperatorTok{=}\NormalTok{ @}\OperatorTok{\{}\NormalTok{key}\OperatorTok{=}\StringTok{"value"}\OperatorTok{\}}
\FunctionTok{Write{-}Output} \VariableTok{$dict}\OperatorTok{.}\FunctionTok{key}
\end{Highlighting}
\end{Shaded}
\item
  List data structures

\begin{Shaded}
\begin{Highlighting}[]
\VariableTok{list}\OperatorTok{=}\VariableTok{(}\NormalTok{1 2 3}\VariableTok{)}
\BuiltInTok{echo} \VariableTok{$\{list}\OperatorTok{[}\DecValTok{0}\OperatorTok{]}\VariableTok{\}}
\end{Highlighting}
\end{Shaded}

\begin{Shaded}
\begin{Highlighting}[]
\BuiltInTok{list} \OperatorTok{=}\NormalTok{ [}\DecValTok{1}\NormalTok{, }\DecValTok{2}\NormalTok{, }\DecValTok{3}\NormalTok{]}
\BuiltInTok{print}\NormalTok{(}\BuiltInTok{list}\NormalTok{[}\DecValTok{0}\NormalTok{])}
\end{Highlighting}
\end{Shaded}

\begin{Shaded}
\begin{Highlighting}[]
\VariableTok{$list} \OperatorTok{=}\NormalTok{ 1}\OperatorTok{,}\NormalTok{ 2}\OperatorTok{,}\NormalTok{ 3}
\FunctionTok{Write{-}Output} \VariableTok{$list}\OperatorTok{[}\NormalTok{0}\OperatorTok{]}
\end{Highlighting}
\end{Shaded}
\end{itemize}

Only extremely common operations will be introduced:

\begin{itemize}
\tightlist
\item
  \texttt{ls}, \texttt{Get-ChildItem}, and \texttt{Path.iterdir()}
\item
  \texttt{cd}, \texttt{Set-Location}, and \texttt{os.chdir()}
\item
  \texttt{mkdir}, \texttt{New-Item}, and \texttt{Path.mkdir()}
\item
  \texttt{rm}, \texttt{Remove-Item}, and \texttt{Path.unlink()}
\item
  \texttt{cat}, \texttt{Get-Content}, and \texttt{Path.read\_text()}
\item
  \texttt{echo}, \texttt{Write-Output}, and \texttt{print()}
\item
  \texttt{grep}, \texttt{Select-String}, and \texttt{re.search()}
\item
  \texttt{sed}, \texttt{Select-String}, and \texttt{re.sub()}
\item
  \texttt{awk}, \texttt{Select-String}, and \texttt{re.split()}
\item
  \texttt{wc}, \texttt{Measure-Object}, and \texttt{len()}
\end{itemize}

You will be invited to participate in srcipting project of CyberPatriot,
where we will build a script that will automate the process of setting
up a secure baseline for Ubuntu, Windows, and Windows Server.

\hypertarget{acl-access-control-list-2102023}{%
\paragraph{ACL (access control list)
(2/10/2023)}\label{acl-access-control-list-2102023}}

In this seminar, you will gain a comprehensive understanding of the ACL
(access control list) feature in Windows and how to effectively use it
to secure your system. One of the key tools you will learn about is
active directory, which allows for systematic management of ACLs, user
accounts, and group memberships across multiple systems with a single
source of truth.

You will also learn about the principle of least privilege and its
importance in securing your system. Topics such as DCom, account policy,
user policy, and group policy will be covered in the user configuration
section. As an example, Autoplay will be disabled. Additionally, you
will learn about password complexity requirements and how to set local
and domain password policies. As an optional topic, you may also learn
about how to set Cisco password policies. ACLs in Linux is already
covered at Section~\ref{sec-harden-ubuntu-fedora} and shall not be
covered in this seminar.

By the end of this seminar, you will be equipped with the knowledge and
skills needed to effectively secure your system using ACLs and other
related tools and techniques.

\hypertarget{common-web-services-332023}{%
\paragraph{Common Web Services
(3/3/2023)}\label{common-web-services-332023}}

In this seminar, you will learn about Cisco networking technology and
its applications in Internet-based systems. You will be introduced to
concepts such as interfaces, IP addressing, VLANs, and Gigabytes
Ethernet interfaces. You will also learn about routing protocols,
specifically OSPF, and their role in managing and maintaining
Internet-based systems. This is considered as an intermediate-level
review of networking concepts.

\begin{Shaded}
\begin{Highlighting}[]
\NormalTok{enable}
\NormalTok{configure terminal}
\NormalTok{interface gigabitethernet 0/0}
\NormalTok{ip address 192.168.10.1 255.255.255.0}
\NormalTok{no shutdown}
\NormalTok{exit}
\end{Highlighting}
\end{Shaded}

\begin{Shaded}
\begin{Highlighting}[]
\NormalTok{enable}
\NormalTok{configure terminal}
\NormalTok{router ospf 1}
\NormalTok{network}
\end{Highlighting}
\end{Shaded}

You will be introduced to the concepts of planning and managing
Internet-based systems. The focus will be on Linux services, with
specific emphasis on Nginx, Apache, and PHP. You will learn how to
configure and optimize these services for optimal performance and
security.

\begin{Shaded}
\begin{Highlighting}[]
\FunctionTok{sudo}\NormalTok{ apt install nginx apache php{-}fpm}
\end{Highlighting}
\end{Shaded}

\begin{Shaded}
\begin{Highlighting}[]
\FunctionTok{sudo}\NormalTok{ systemctl enable }\AttributeTok{{-}{-}now}\NormalTok{ nginx apache php{-}fpm}
\end{Highlighting}
\end{Shaded}

Throughout the seminar, you will participate in hands-on activities that
will allow you to apply the concepts you have learned to real-world
scenarios. You will start with Cisco, learning about its interfaces and
routing, then move on to Linux services, learning about Nginx, Apache,
and PHP.

\hypertarget{network-security-3102023}{%
\paragraph{Network Security
(3/10/2023)}\label{network-security-3102023}}

In this seminar, you will delve into the topic of security in network
systems, with a particular focus on Cisco. You will learn about a
variety of security-related features and protocols, including SSH and
RSA for secure telecommunication, NAT for network address translation,
Telnet for remote access, and the console port for physical access
control.

You will also learn about Cisco's AAA (Authentication, Authorization,
and Accounting), which provides a centralized way to manage access
control and user authentication. Additionally, you will learn about
Cisco's audit feature and its role in monitoring network activity and
detecting security breaches. Finally, VPN (Virtual Private Network) will
be covered, which allows for secure communication over public networks.

If there is extra time, SSH in Linux will be covered, e.g., generate
public and private keys, and configure SSH server and client.

\begin{Shaded}
\begin{Highlighting}[]
\NormalTok{enable}
\NormalTok{configure terminal}
\NormalTok{ip domain{-}name anish.com}
\NormalTok{crypto key generate rsa}
\NormalTok{ip ssh version 2}
\NormalTok{username admin privilege 15 secret admin}
\NormalTok{line vty 0 4}
\NormalTok{transport input ssh}
\NormalTok{login local}
\end{Highlighting}
\end{Shaded}

\begin{Shaded}
\begin{Highlighting}[]
\NormalTok{enable}
\NormalTok{configure terminal}
\NormalTok{aaa new{-}model}
\NormalTok{aaa authentication login default local}
\NormalTok{radius{-}server host 192.168.1.1 key cisco}
\NormalTok{line vty 0 4}
\NormalTok{login authentication default}
\end{Highlighting}
\end{Shaded}

\begin{Shaded}
\begin{Highlighting}[]
\NormalTok{crypto isakmp policy 10}
\NormalTok{encryption aes}
\NormalTok{hash sha}
\NormalTok{authentication pre{-}share}
\end{Highlighting}
\end{Shaded}

\hypertarget{common-service-configuration-3242023}{%
\paragraph{Common Service Configuration
(3/24/2023)}\label{common-service-configuration-3242023}}

In this seminar, you will learn about common services and their
configurations in Linux and Windows systems. You will begin by learning
about file services, specifically FTP (Vsftpd) and NFS, and how to set
them up and configure them for optimal performance and security. You
will also learn about Samba, a service that allows Linux systems to
share files with Windows systems and vice versa.

Additionally, you will learn about database management, specifically
MySQL/MariaDB and basic SQL. We will not focus on CRUD operations, but
rather on how to set up a database server and configure it for optimal
security.

\begin{Shaded}
\begin{Highlighting}[]
\KeywordTok{CREATE}\NormalTok{ LOGIN }\KeywordTok{admin} \KeywordTok{WITH} \KeywordTok{PASSWORD} \OperatorTok{=} \StringTok{\textquotesingle{}admin\textquotesingle{}}\NormalTok{;}
\KeywordTok{CREATE} \FunctionTok{USER} \KeywordTok{admin} \ControlFlowTok{FOR}\NormalTok{ LOGIN }\KeywordTok{admin}\NormalTok{;}
\KeywordTok{CREATE} \KeywordTok{DATABASE}\NormalTok{ test;}
\KeywordTok{GRANT} \KeywordTok{ALL} \KeywordTok{ON}\NormalTok{ test.}\OperatorTok{*} \KeywordTok{TO} \KeywordTok{admin}\NormalTok{;}
\end{Highlighting}
\end{Shaded}

Optionally, we may cover \texttt{choco} and \texttt{scoop} for Windows.

\hypertarget{forensics-guide-3312023}{%
\paragraph{Forensics Guide (3/31/2023)}\label{forensics-guide-3312023}}

Today is the start of CS club officer applications. If you are
interested in applying, please fill out the form. In this seminar, we
will explore forensics questions. For Windows, Process Explorer, Process
Monitor, Show hidde, TCP View and Treesize free will be introduced. The
``Get-Service'' command can be used to view and manage services on a
Windows system.For Linux, \texttt{crontab}, \texttt{services}, and
\texttt{systemd} will be introduced Crontab is a Linux utility that is
used to schedule and run recurring tasks, while services and systemd are
used to manage and configure running processes and services. Wireshark
and Netstat can be used to analyze network traffic and connections,
regardless of the operating system. We will also learn about
\texttt{tcpdump}, which is a command-line tool that can be used to
capture and analyze network traffic.

\hypertarget{operating-systems-4142023}{%
\paragraph{Operating Systems
(4/14/2023)}\label{operating-systems-4142023}}

Operating systems are responsible for managing the resources of a
computer system. In this seminar, you will learn about the basic
concepts of operating systems, including processes, threads, user-space,
kernal space, bootloader, kernal parameters, system calls, and virtual
file systems.

First, we will look at the grub bootloader, which is responsible for
loading the Linux kernel and initializing the operating system. We will
learn how to configure and troubleshoot grub. We will also discuss the
procedure of booting a Linux system:

\begin{enumerate}
\def\labelenumi{\arabic{enumi}.}
\tightlist
\item
  BIOS performs POST upon power-on
\item
  BIOS loads bootable device, e.g., boot loader at MBR
\item
  GRUB is loaded and executed
\item
  GRUB loads the initramfs, which is a temporary file system
\item
  Initramfs loads the kernel
\item
  Kernel loads init, which is the first process to run. Systemd is the
  default init system in modern Linux distributions.
\end{enumerate}

Next, we will delve into \texttt{sysctl}, a tool that allows us to
configure kernel parameters at runtime. We will learn how to use sysctl
to optimize the performance and security of our Linux systems. For
example, IPv4 assassination attack can be prevented by setting
\texttt{net.ipv4.tcp\_rfc1337} to 1.

After that, we will learn about system calls, which are the interface
between user-space and kernal-space. We will learn how to use
\texttt{strace} to trace system calls and \texttt{ltrace} to trace
library calls. We will also learn how to use \texttt{gdb} to debug
system calls.

Finally, we will explore LVM, RAID, VFS, and the mount command, which
are all used to manage and organize storage in Linux. In addition, we
will take a deep dive into the structure of the Ext4 file system, and
learn about \texttt{ext4undelete} and \texttt{sleuthkit}, which are
tools that can be used to recover deleted files.

\hypertarget{basic-setup-4212023}{%
\paragraph{Basic Setup (4/21/2023)}\label{basic-setup-4212023}}

In this seminar, we will summarize and review the concepts we have
learned this year. We will create a checklist of things to do when a new
image had been delivered to us.

For linux, here are some samples: - configuring OpenSSH on Linux -
updating packages with Apt Windows: - enabling interactive login and
Ctrl+Alt+Del on Windows - creating backup and restore points - securing
drives For all of them: - installing VMWare tools - adjusting screen
resolution - cleaning user accounts and removing unwanted software -
protecting against malware with anti-virus.

We will also learn about the diversity of Linux distributions and how to
choose the right one for your needs (basically, how to be a good Linux
user). We will cover a few topics below:

\begin{itemize}
\tightlist
\item
  Package managers, \texttt{apt}, \texttt{dnf}, and \texttt{pacman}
  (Ubuntu, Fedora, and Arch)
\item
  GUI development frameworks, \texttt{qt}, \texttt{gtk}, and \texttt{wx}
\item
  Translation layers, \texttt{wine}, \texttt{proton}, and
  \texttt{darling}
\item
  Modern cloud computing and containerization, \texttt{docker},
  \texttt{kubernetes}, and \texttt{podman}
\end{itemize}

\begin{figure}

{\centering \includegraphics[width=\textwidth,height=1.5in]{images/kde.png}

}

\caption{Kde}

\end{figure}

\begin{figure}

{\centering \includegraphics[width=\textwidth,height=1.5in]{images/gnome.png}

}

\caption{Gnome}

\end{figure}

\hypertarget{chill-4282023}{%
\paragraph{Chill (4/28/2023)}\label{chill-4282023}}

End of year celebration. Discuss summer plans. Officer announcements.
Plans for next year.



\end{document}
